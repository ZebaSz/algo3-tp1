\section{Descripción del problema}

	\subsection{Descripción general}
	El problema propuesto pide que se encuentre la menor cantidad "elementos sin pintar" posibles para una lista de números enteros siguiendo el siguiente criterio:
	
	\begin{itemize}
		\item Todos los elementos pueden solo estar pintados de azul, de rojo, o estar sin pintar (mutuamente exclusivos).
		\item Todos los elementos pintados de rojo deben estar en orden \textit{estrictamente creciente}.
		\item Todos los elementos pintados de azul deben estar en orden \textit{estrictamente decreciente}.
	\end{itemize}

	El problema, para una lista dada de números, se considera que tiene una resolución \textbf{óptima} si existe una manera de pintar todos los elementos de dicha lista ya sea de rojo o de azul, es decir, no queda ningún elemento sin pintar.

	\subsection{Ejemplos provistos por la cátedra}
	
	Para estos ejemplos, se denotará debajo de cada número con una letra si el mismo se pinta de \textbf{R}ojo, \textbf{A}zul o \textbf{N}o pintado.

	\begin{enumerate}
		\item Ejemplo con solución óptima

		\begin{tabular}{ c c c c c c c c }
			0 & 7 & 1 & 2 & 2 & 1 & 5 & 0 \\
			R & A & R & R & A & A & R & A \\
		\end{tabular}

		Resultado: 0 (óptima)

		En este ejemplo se pueden encontrar subsecuencias que cumplan los criterios mencionados: una de elementos rojos (0 1 2 5) y otra de elementos azules (7 2 1 0).

		\item Ejemplo sin solución óptima

		\begin{tabular}{ c c c c c c c c c c c c }
			3 & 11 & 0 & 1 & 3 & 5 & 2 & 4 & 1 & 0 & 9 & 3 \\
			N & A & R & R & R & A & A & R & A & A & R & N \\
		\end{tabular}

		Resultado: 2

		En este ejemplo no existe un par de subsecuencias disjuntas que satisfagan todos los criterios, por lo que el primer y el último 3 son ignorados para maximizar la cantidad de elementos pintados.
	\end{enumerate}

	\subsection{Otros ejemplos}

	\begin{enumerate}
		\item Ejemplo con único número

		\begin{tabular}{ c c c c c c c c }
			0 & 0 & 0 & 0 & 0 & 0 & 0 & ... \\
			R & A & N & N & N & N & N & ... \\
		\end{tabular}

		Resultado: longitud de la lista - 2

		Para cualquier longitud de lista, si la misma solo contiene elementos idénticos, solo pueden pintarse 2 elementos (ya que los rojos y los azules deben ser \underline{estrictamente} crecientes y decrecientes respectivamente).

		En general, sea $l$ una lista cualquiera y sea $f(l)$ su solución,  $0 \leq f(l) \leq max(|l| - 2, 0)$ (en otras palabras, si la lista tiene más de dos elementos, la solución siempre incluye por lo menos dos elementos pintados).

		\item Ejemplo con un solo color

		\begin{tabular}{ c c c c c c c c }
			0 & 1 & 2 & 3 & 4 & 5 & 6 & ... \\
			R & R & R & R & R & R & R & ... \\
		\end{tabular}

		Resultado: 0 (óptima)

		Para cualquier longitud de lista, si la misma es estrictamente creciente todos los elementos pueden pintarse del mismo color. El problema no requiere que haya elementos pintados de ambos colores.

		Esta regla aplica para cualquier secuencia $l$ con elementos $l_1, l_2, ..., l_n$ tales que

		\begin{center}
			$(\forall i, j \leftarrow i < j)(l_i < l_j)$
		\end{center}

		es decir, para toda lista estrictamente creciente. La regla vale para el caso opuesto (la lista es estrictamente decreciente), pintando de azul en lugar de rojo.

		Para este ejemplo también existe una cantidad de soluciones igual a la longitud de la lista en las que una de las posiciones está pintada del otro color (solo uno puede estar pintado de ese color).
	\end{enumerate}
